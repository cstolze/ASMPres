\documentclass{beamer}
\mode<presentation>
{
  \usetheme{Warsaw}


  \setbeamercovered{transparent}

}
\usepackage[french]{babel}
\usepackage[utf8]{inputenc}
\usepackage{times}
\usepackage[T1]{fontenc}
\usepackage{graphicx}
\usepackage{verbatim}

\title[Compilateur et débogueur] % (optional, use only with long paper titles)
{Compilateur et débogueur}

\author[Claude Stolze, Pierre Donat-Bouillud, Maël Lansade]{Claude Stolze , Pierre Donat-Bouillud \and Maël Lansade}

\institute[ENS Cachan - Antenne de Bretagne] 

\begin{document}
\begin{frame}
  \titlepage
\end{frame}

%\begin{frame}{Plan}
%  \tableofcontents
%\end{frame}

\section{Introduction}

\begin{frame}
	\frametitle{Suite logiciel pour la programmation en langage d'assemblage}
	\begin{itemize}
	\item codé en C++ (ou en C+ $\dots$)
	\item git et github
	\item trois sous-modules :
		\begin{itemize}
		\item assembleur
		\item simulateur
		\item debugger
		\end{itemize}
	\end{itemize}
\end{frame}

\section{Compilateur}
\subsection{Présentation}
\begin{frame}[fragile]
\begin{verbatim}
.BSS // ou .DATA
	label1 : 4		// taille du bloc de données.
	label2 : 1+1

.CODE
	label3 : add r0 r0 r0
\end{verbatim}
\end{frame}

\begin{frame}[fragile]
\frametitle{Expressions arithmétiques}
Opérations autorisées~: addition, soustraction, multiplication, division, modulo, exposant.
\begin{verbatim}
.BSS
label1 : label2 // illégal
label2 : (1+1)*2^(4/2) % 3

.CODE
	jmp label1 // légal
\end{verbatim}
\end{frame}

\subsection{Structure du programme}
\begin{frame}
\frametitle{Grammaire}
Le langage accepté par le programme est défini par une grammaire hors contexte.
Par exemple, on a la règle suivante :
\[ S \rightarrow .BSS\ \backslash n\ lignesdonnees\ S\ |\ .CODE\ \backslash n\ lignescodes\ S\ |\ \epsilon \]
\end{frame}

\begin{frame}
L'analyse se déroule en trois parties~:
\begin{itemize}
\item Lecture du fichier~: on lit le fichier en supprimant les commentaires~;
\item Analyse lexicale~: on cherche à reconnaitre des mots~;
\item Analyse syntaxique~: on regarde si le texte satisfait la grammaire.
\item Analyse sémantique~: étape finale, où on regarde si le code a un sens.
\end{itemize}
\end{frame}

\begin{frame}
Structures utilisées dans le programme~:
\begin{itemize}
\item arbres pour représenter les expressions arithmétiques~;
\item table de hachage pour associer des valeurs aux symboles~;
\item liste d'instruction.
\end{itemize}
\end{frame}

\section{Simulateur}
\subsection{Présentation}
\begin{frame}
Le simulateur prend en entrée un code binaire, et l'exécute.
\end{frame}

\subsection{Fonctionnement}
\begin{frame}
\frametitle{Mémoire}
	Il y a deux mémoires : pour le programme et pour les données.
	
	La mémoire de programme est initialisée à la lecture du fichier d'entrée.
\end{frame}

\begin{frame}
\frametitle{Unité de contrôle}
L'unité de contrôle lit une instruction, la décode, et l'exécute (et peut recommencer \emph{ad lib}).
\end{frame}

\section{Débogueur}

\subsection{Commandes}

\begin{frame}
	\frametitle{La référence : gdb}
	GDB : GNU debugger
	Etudes des commandes de gdb pour mettre au point les commandes.
\end{frame}

\begin{frame}[fragile]
	\frametitle{Commandes}
	
	\begin{block}{Types de commande}
		\begin{itemize}
		\item points d'arrêt
		\item affichage de données internes (ou modifications)
		\item persistance de l'état du simulateur
		\item exécution du programme
		\item pile
		\item commandes internes
		\item chronomètre
		\end{itemize}
	\end{block}
	
	\begin{block}{Syntaxe}
		\verb| commande options |
	\end{block}
	
\end{frame}

\begin{frame}[fragile]
	\frametitle{Points d'arrêt}
	\begin{block}{Syntaxe}
		\begin{itemize}
		\item \verb|breakpoint line n|
		\item \verb|breakpoint address addr [offset]|
		\item \verb|breakpoint label lab|
		\item \verb|breakpoint port pt [mode] [val]|
		\item \verb|breakpoint instr i|
		\end{itemize}
	\end{block}
	
	\begin{block}{Symboles}
		Faire un breakpoint sur un label $\longrightarrow$ charger les symboles.
		
		\verb|symbol file|
	\end{block}
\end{frame}

\begin{frame}[fragile]
	\frametitle{Affichage/modifications d'informations sur le simulateur}
	\begin{block}{Syntaxe}
		\begin{itemize}
		\item \verb|display register [reg]|
		\item \verb|display address add [offset] |
		\item \verb|dump address add offset fileName|
		\item \verb|find seq|
		\item \verb|find-next|
		\item \verb|print texte|
		\item \verb|write register reg val|
		\item \verb|write address addr [offset] val|
		\item \verb|write port [in/out] p val|
		\end{itemize}
	\end{block}
\end{frame}
	
\begin{frame}[fragile]
	\frametitle{Persistance de l'état du simulateur}
	\begin{block}{Syntaxe}
		\begin{itemize}
		\item \verb| record file|
		\item \verb| restore file |
		\end{itemize}
	\end{block}
\end{frame}

\begin{frame}[fragile]
	\frametitle{Exécution du programme}
	\begin{block}{Syntaxe}
		\begin{itemize}
		\item \verb|run |
		\item \verb|next |
		\item \verb|continue |
		\item \verb|step n|
		\item \verb|exit | ou \verb|Ctrl+d|
		\item \verb|until |
		\end{itemize}
	\end{block}
\end{frame}

\subsection{Fonctionnement}

\begin{frame}[fragile]
	\frametitle{Architecture}
	Associer un simulateur au debugger :

		\begin{itemize}
	\item \verb|Debugger::attach(CPU* cpu)|
	\item \verb|Debugger::debug()|
	\end{itemize}
	
	\begin{block}{Simulateur seulement:}
		\begin{verbatim}
			
    control_unit->read_inst();
    control_unit->decode_inst();
    control_unit->execute_inst();	
		\end{verbatim}
	\end{block}

\end{frame}

\begin{frame}[fragile]
	\frametitle{Architecture}
	Associer un simulateur au debugger :
	\begin{block}{Interaction avec le simulateur :}
	\begin{verbatim}
        bool cont = true;
        while (sim->getBus_inst() != -1 && cont)
        {
            sim->run();
            cont = interact();
            
        }
        interf.answer("Fin de l'exécution.");
        
        return cont;
		\end{verbatim}
	\end{block}
\end{frame}

\begin{frame}[fragile]
	\frametitle{Interface}
	\verb|commandInterface| virtuelle.

	Plusieurs sorties possibles \footnote{Seule la sortie console est implémentée} :
	\begin{itemize}
	\item console
	\item graphique
	\item réseau
	\end{itemize}
\end{frame}

\begin{frame}
	\frametitle{Analyse des commandes}
	
\end{frame}

\begin{frame}
	\frametitle{Gestion des erreurs}
\end{frame}

\end{document}