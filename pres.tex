\documentclass{beamer}
\mode<presentation>
{
  \usetheme{Warsaw}


  \setbeamercovered{transparent}

}
\usepackage[french]{babel}
\usepackage[utf8]{inputenc}
\usepackage{times}
\usepackage[T1]{fontenc}
\usepackage{graphicx}
\usepackage{verbatim}

\title[Compilateur et débogueur] % (optional, use only with long paper titles)
{Compilateur et débogueur}

\author[Claude Stolze, Pierre Donat-Bouillud, Maël Lansade]{Claude Stolze , Pierre Donat-Bouillud \and Maël Lansade}

\institute[ENS Cachan - Antenne de Bretagne] 

\begin{document}
\begin{frame}
  \titlepage
\end{frame}

\begin{frame}{Plan}
  \tableofcontents
\end{frame}

\section{Compilateur}
\subsection{Présentation}
\begin{frame}[fragile]
\begin{verbatim}
.BSS // ou .DATA
	label1 : 4		// taille du bloc de données.
	label2 : 1+1

.CODE
	label3 : add r0 r0 r0
\end{verbatim}
\end{frame}

\begin{frame}[fragile]
\frametitle{Expressions arithmétiques}
Opérations autorisées~: addition, soustraction, multiplication, division, modulo, exposant.
\begin{verbatim}
.BSS
label1 : label2 // illégal
label2 : (1+1)*2^(4/2) % 3

.CODE
	jmp label1 // légal
\end{verbatim}
\end{frame}

\subsection{Structure du programme}
\begin{frame}
\frametitle{Grammaire}
Le langage accepté par le programme est défini par une grammaire hors contexte.
Par exemple, on a la règle suivante :
\[ S \rightarrow .BSS\ \backslash n\ lignesdonnees\ S\ |\ .CODE\ \backslash n\ lignescodes\ S\ |\ \epsilon \]
\end{frame}

\begin{frame}
L'analyse se déroule en trois parties~:
\begin{itemize}
\item Lecture du fichier~: on lit le fichier en supprimant les commentaires~;
\item Analyse lexicale~: on cherche à reconnaitre des mots~;
\item Analyse syntaxique~: on regarde si le texte satisfait la grammaire.
\item Analyse sémantique~: étape finale, où on regarde si le code a un sens.
\end{itemize}
\end{frame}

\begin{frame}
Structures utilisées dans le programme~:
\begin{itemize}
\item arbres pour représenter les expressions arithmétiques~;
\item table de hachage pour associer des valeurs aux symboles~;
\item liste d'instruction.
\end{itemize}
\end{frame}

\section{Simulateur}
\subsection{Présentation}
\begin{frame}
Le simulateur prend en entrée un code binaire, et l'exécute.
\end{frame}

\subsection{Fonctionnement}
\begin{frame}
\frametitle{Mémoire}
	Il y a deux mémoires : pour le programme et pour les données.
	
	La mémoire de programme est initialisée à la lecture du fichier d'entrée.
\end{frame}

\begin{frame}
\frametitle{Unité de contrôle}
L'unité de contrôle lit une instruction, la décode, et l'exécute (et peut recommencer \emph{ad lib}).
\end{frame}

\section{Débogueur}
\subsection{Présentation}

\end{document}